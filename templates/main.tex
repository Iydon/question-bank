% !Mode:: "TeX:UTF-8"
% !TEX program  = xelatex
\newcommand{\学校全称}{南方科技大学}
\newcommand{\学校英文}{Southern University of Science and Technology}
\newcommand{\学校简称}{SUSTech}
\newcommand{\学年学期}{2019--2020学年秋季学期}
\newcommand{\科目全称}{计算金融}
\newcommand{\科目代码}{MA216}
\newcommand{\试卷类别}{A}
\newcommand{\考试方式}{闭卷}
\newcommand{\打印纸张}{a4paper} % 有待处理 a3paper 的结构问题
\newcommand{\展示答案}{true} % true or false



\documentclass{exam}
	% 使用 XeLaTeX 或者 LuaLaTeX 编译.
	\usepackage{ifxetex,ifluatex}
	\usepackage{ifthen}
		\ifthenelse{\boolean{xetex}\OR\boolean{luatex}}{}{
			\errmessage{You Should Use XeLaTeX or LuaLaTeX To Compile.}}
	% 配置选项.
	\usepackage{geometry}
		\geometry{layout=\打印纸张}
	\usepackage{xstring}
		\IfStrEq{\展示答案}{true}{\printanswers}{}
	% 使用中文, 颜色, 数学符号等.
	\usepackage{ctex,xcolor,mathtools,amssymb}
	% Exam 文类选项
		% 提示语配置
		\renewcommand{\solutiontitle}{\noindent\textbf{答案:}}
		% 正确答案样式
		\CorrectChoiceEmphasis{\color{black}\bfseries}
		% 分数计量
		\pointpoints{分}{分} % 单复数
		\bonuspointpoints{分 (附加)}{分 (附加)} % 附加分单复数
		% 宏与环境
		\newenvironment{不定项选择}%
			{\begin{oneparchoices}}%
			{\end{oneparchoices}}
		\newenvironment{纵向不定项选择}%
			{\begin{choices}}%
			{\end{choices}}
		\newcommand{\填空}[1]{\fillin[#1]}
		\newcommand{\判断正误}[1]{\fillin[#1][0.25in]}
		\newenvironment{解析}%
			{\renewcommand{\solutiontitle}{\noindent\textbf{解析:}}\begin{solutionorbox}}
			{\renewcommand{\solutiontitle}{\noindent\textbf{答案:}}\end{solutionorbox}}
	% 测试文档
	\usepackage{zhlipsum}
\begin{document}
\begin{questions}
	% 不定项选择题
	\question[5] 本题为不定项选择.\par
	\begin{不定项选择}
		\choice 错误答案
		\CorrectChoice 正确答案
		\choice 错误答案
		\CorrectChoice 正确答案
		\choice 错误答案
	\end{不定项选择}
	\begin{解析}
		这里为解析部分: \zhlipsum[1]
	\end{解析}
	\question[5] 本题为纵向不定项选择.
	\begin{纵向不定项选择}
		\choice 错误答案
		\CorrectChoice 正确答案
		\choice 错误答案
		\CorrectChoice 正确答案
		\choice 错误答案
	\end{纵向不定项选择}
	\begin{解析}
		这里为解析部分: \zhlipsum[1]
	\end{解析}
	% 填空题
	\question[5] 本题为填空题, 此空为\填空{正确答案}.
	\begin{解析}
		这里为解析部分: \zhlipsum[1]
	\end{解析}
	\question[5] \判断正误{T} 本题为判断正误题.
	\begin{解析}
		这里为解析部分: \zhlipsum[1]
	\end{解析}
	% 简答题
\end{questions}
\end{document}
