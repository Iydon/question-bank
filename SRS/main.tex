% !Mode:: "TeX:UTF-8"
% !TEX program  = xelatex
\newcommand{\项目名称}{基于 \TeX\ 的题库整理与试卷构建系统}
\newcommand{\项目组织}{\texttt{Mathlang} 组织}
\newcommand{\项目版本}{0.1}
\newcommand{\项目链接}{https://github.com/Iydon/question-bank}
\newcommand{\项目问题}{\项目链接/issues}



\documentclass{ctexart}
    % Packages
    %% 超链接
    \usepackage{hyperref}
        \hypersetup{
            pdftitle={Software Requirement Specification},
            pdfauthor={Iydon Liang},
            pdfcreator={Microsoft® Word 2017},
            pdfproducer={Microsoft® Word 2017},
            colorlinks=true,
            linkcolor=black,
        }
    %% 页面布局
    \usepackage[margin=1in]{geometry}
    %% 表格及颜色
    \usepackage{tabularx}
    \usepackage{booktabs}
    \usepackage[table,xcdraw]{xcolor}
    %% 行间列表
    \usepackage[inline]{enumitem}
    %% LaTeX logo
    \usepackage{hologo}
    %% 参考文献
    \usepackage[style=gb7714-2015]{biblatex}
        \addbibresource{ref.bib}
    %% 代码
    \usepackage{minted}
    % Nouns
    \newcommand{\github}{\texttt{GitHub}}
    \newcommand{\python}{\texttt{Python}}
    \newcommand{\pylatex}{\texttt{PyLaTeX}}
    \newcommand{\pythondocx}{\python\texttt{-docx}}
    \newcommand{\json}{\texttt{Json}}
    \newcommand{\sql}{\texttt{SQL}}
    \newcommand{\sqlalchemy}{\sql\texttt{Alchemy}}
    \newcommand{\wikipedia}{\texttt{Wikipedia}}
    \newcommand{\readthedocs}{\texttt{Read the Docs}}
    \newcommand{\pypi}{\texttt{PyPi}}
    \newcommand{\requirements}{\texttt{requirements.txt}}
    \newcommand{\pipfile}{\texttt{Pipfile}}
    \newcommand{\api}{\texttt{API}}
    \newcommand{\gui}{\texttt{GUI}}
    \newcommand{\pepeight}{\texttt{PEP 8}}
    \newcommand{\click}{\texttt{click}}
    \newcommand{\nginx}{\texttt{Nginx}}
    \newcommand{\https}{\texttt{HTTPS}}
    % Information
    \title{软件需求说明: \项目名称\ (\项目版本\ 版本)}
    \author{\项目组织}
    \date{\today}
\begin{document}
\maketitle\tableofcontents\clearpage


\section{修订历史}\label{S:revision-history}
\begin{center}
    \begin{tabularx}{\textwidth}{ccXc}
        \toprule
            维护者 & 时间 & 变更日志 & 版本 \\
        \midrule
        \rowcolor[HTML]{EFEFEF}
            梁钰栋 & 2019/11/19 & \begin{itemize*}[itemjoin={\newline}]
                    \item 初步定稿软件需求说明 (本地版本)
                    \item 在 \href{\项目链接}{\github} 新建仓库托管代码
                \end{itemize*} & 0.1 \\
        \bottomrule
    \end{tabularx}
\end{center}



\section{简介}\label{S:introduction}
\subsection{项目目的}
\项目名称\ (下称本项目) 旨在将试卷的格式与数据分离 (即分离为前端, 后端及数据库): 使用 \json\ 数据格式或数据库 (如 \sql) 来存储题目数据; 使用 \TeX\ 系统对题目进行排版, 数据与版式之间通过 \python\ 编程语言操作数据库以生成 \TeX\ 源代码.

本项目的思想来自于使用特定工具解决特定的问题, 同时从工程的角度审视问题, 寻求易于拓展的解决方案:
\begin{enumerate*}[label=\textbf{(\arabic*)}]
    \item \TeX\ 擅长排版, 但用作运算 (如计数器的数学运算等) 的用户接口较弱, 而支持嵌入式脚本的原生引擎 \hologo{LuaLaTeX} 则需要更复杂的逻辑实现, 不利于以后对程序进行改进与维护;
    \item \python\ 为``胶水语言'', 功能库众多, 可以根据特定需求做特定事情 (比如从数据库筛选题目, 制作网页应用等), 但是排版所用库 (例如 \pylatex, \pythondocx) 提供的功能不如 \TeX\ 丰富且学习成本较大;
    \item 数据库仅用来存储数据与筛选数据, 所以在本项目中不能单独使用. 而 \python\ 存在 ORM (Object Relational Mapping, 例如 \sqlalchemy\cite{sqlalchemy}), 可用来操作数据库. 但是本项目一开始是作为本地版本开发, 数据应追求可读性, 所以使用纯文本记录数据 (例如数据格式 \json) 可以更方便地修改数据.
\end{enumerate*}


\subsection{文档约定}
``\项目名称'' 的文档 (下称本文档) 参考 \wikipedia\cite{wiki2019srs} 与 \github\cite{jean2016latex} 的提示进行编写, 其中文档采用 \TeX\ 系统进行排版, 除数据外的代码等部分均托管在 \href{\项目链接}{\github} 上. 需要注意的是, 本文档的源代码中包含大量的交叉引用, 如无必要, 请勿对文件进行复制粘贴, 直接转载到其他网站上.
\begin{center}
    \begin{tabularx}{\textwidth}{cX}
        \toprule
            名词 & 解释 \\
        \midrule
        \rowcolor[HTML]{EFEFEF}
            本地版本 & 使用 \TeX, \python, \json\ 的版本, 旨在在本地进行题库的管理, 后期可能会在在线版本中提供题库上传与导出的功能, 用以管理个人的题库. \\
            在线版本 & 使用 \TeX, \python, \sql\ 的版本, 旨在在网页进行题库的管理, 后期可能会在在线版本中提供题库上传与导出的功能, 用以管理团体的题库, 同时在线合作完善题库. \\
        \rowcolor[HTML]{EFEFEF}
            版本控制 & 对软件开发过程中各种程序代码, 配置文件及说明文档等文件变更的管理. \\
            \TeX & 本项目使用的 \TeX\ 发行版为 \TeX Live 2019, 除此之外的所有发行版 (例如 C\TeX) 均不保证模板的正常运行, 更多资料请参考 \href{https://tex.readthedocs.io}{\readthedocs}. \\
        \rowcolor[HTML]{EFEFEF}
            \sqlalchemy & 托管在 \href{https://pypi.org/project/SQLAlchemy/}{\pypi} 上的 \python\ 库, 具体版本需要见程序中的依赖声明 (\requirements, \pipfile\ 等), 此外在本地版本中不会使用. \\
        \bottomrule
    \end{tabularx}
\end{center}


\subsection{目标读者与阅读建议}
本文档的目标读者为本项目的开发人员及在初期使用本项目的用户. 以下将按照不同的读者类型分别列举其阅读建议. 此外, 如果您不属于此处列举的目标读者, 但是仍想获取阅读建议, 请及时提出 \href{\项目问题}{Issue}, 我们会在注意到后及时补充至本文档中.
\begin{center}
    \begin{tabularx}{\textwidth}{cX}
        \toprule
            读者类型 & 阅读建议 \\
        \midrule
        \rowcolor[HTML]{EFEFEF}
            开发人员 & 建议着重阅读第~\ref{S:external-interface-requirements}~节, 第~\ref{S:system-features}~节及第~\ref{S:other-nonfunctional-requirements}~节, 如果对本文档存在疑问请及时提出 \href{\项目问题}{Issue}. \\
            初期用户 & 建议浏览第~\ref{S:introduction}~节及第~\ref{S:overall-description}~节, 如果对本文档存在疑问请及时提出 \href{\项目问题}{Issue}. \\
        \bottomrule
    \end{tabularx}
\end{center}


\subsection{项目范围}
本项目重点为题库整理与试卷构建. 在前期项目 \href{https://github.com/Iydon/LaTeX_template/tree/master/Exam}{Exam} 中, 我们完全使用 \TeX\ 编写题库, 通过 \TeX\ 的 \verb|\input| 机制将不同时间的题目分离, 与此同时产生一系列问题:
\begin{enumerate*}[label=\textbf{(\arabic*)}]
    \item 题目看似分离, 但是在根据已有题目来构建新的试卷时, 还需要将 \TeX\ 的冗余部分重新书写一遍;
    \item 为了追求注释部分 \verb|\input| 后仍可以运行, 导言区需要将所有的依赖记录下来, 从而造成性能上的损失;
    \item 前期项目提供众多的 \api\ (尤其解题框), 构建题库的时间跨度较大导致使用 \api\ 不统一.
\end{enumerate*}

而本项目在设计时改进了上述的缺点, 即将格式与数据完全分离. 题库存储的时候不需要考虑 \TeX\ 的格式, 试卷构造的时候会针对不同的题目进行导言区的构造 (耗费时间可以忽略不计, \TeX\ 宏包加载的时间远大于程序运行的时间), 在提高题库自定义化的同时, 更好地管理题库的数据.

如果后期添加了在线版本的支持, 则可以将用户们的题库进行整合, 在标签明确的前提下, 可以快速构建试卷. 同时, 可以邀请一部分与试卷无利害关系的用户, 对试卷的内容解析等进行更好地优化, 同时记录修改的历史记录, 用以量化用户的贡献.


\subsection{参考资料}
\printbibliography[heading=none]



\section{总体描述}\label{S:overall-description}
\subsection{产品总览}
本项目首先开发本地版本. 如果有需求, 则继续开发在线版本 (即套用离线版本, 但使用数据库存储数据, 同时添加类似 \href{https://github.com/Iydon}{\github} 的用户管理); 如果有需求, 则继续开发本地版本的桌面版本, 避免用户命令行编译 (即将本地版本 \api\ 综合起来, 添加 \gui\ 支持).

本项目的简介请参考第~\ref{S:introduction}节. 本地版本主要分为三个部分: \begin{enumerate*}[label=\textbf{(\arabic*)}]
    \item 模板文件, 主要为本地版本的默认模板, 如果想自定义, 可以手动修改提供模板 (目前为硬编码在 \python\ 文件中, 实现初步功能后代码重构会解决此问题);
    \item 题库数据文件, 主要以 \json\ 数据结构存储题目信息, 针对不同的题型设计了不同的键值来存储, 同时继承了相同的键值 (例如题目类型, 描述, 分数, 难度, 解析), 数据模型划分为不定项选择, 不定数填空, 判断正误;
    \item 代码逻辑部分不建议非开发人员修改, 而开发人员不需要阅读此条.
\end{enumerate*}

在线版本除了在线自定义模板文件及题库数据文件外, 增加存储题库, 合作编辑及版本控制的功能, 此处可以参考 \href{https://github.com/Iydon}{\github} 的形式. 主要分为六个部分: \begin{enumerate*}[label=\textbf{(\arabic*)}]
    \item 注册用户;
    \item 创立公有, 私有题库;
    \item 添加合作者至题库;
    \item 上传, 下载题库;
    \item 在线编辑题库 (数据及样式);
    \item 在线生成试卷.
\end{enumerate*}


\subsection{产品功能}
\begin{center}
    \begin{tabularx}{\textwidth}{cX}
        \toprule
            版本 & 功能 \\
        \midrule
        \rowcolor[HTML]{EFEFEF}
            本地版本 & \begin{itemize*}[itemjoin={\newline}]
                    \item 使用 \json\ 数据格式存储题目以构成题库, 纯文本的格式方便用户查看与修改. 但是格式需与在线版本的数据库结构一致, 方便数据库迁移;
                    \item 使用 \TeX\ 系统构建试卷, 提供默认样式, 同时支持用户自定义样式;
                    \item 使用 \python\ 连接数据与格式, 抽象题库的筛选算法;
                    \item 最初提供命令行编译版本, 后期可以提供不同平台的 \gui\ 程序 (使用程序均可跨平台).
                \end{itemize*} \\
            在线版本 & \begin{itemize*}[itemjoin={\newline}]
                    \item 注册新用户, 上传与下载 \json\ 格式的题库, 并且在线对题库进行编辑;
                    \item 根据标签对题库进行筛选, 构造试卷, 同时控制分数及难度;
                    \item 用户管理类似 \href{https://github.com/Iydon}{\github}. 公有题库可以通过用户页进行查看, 私有题库则需要通过分享链接进行修改 (或添加为合作人员), 同时进行版本控制;
                    \item 在线版本的数据库存储需进行加密, 防止后台出现问题.
                \end{itemize*} \\
        \bottomrule
    \end{tabularx}
\end{center}


\subsection{用户类别及其特征}
\begin{center}
    \begin{tabularx}{\textwidth}{cX}
        \toprule
            用户类别 & 特征 \\
        \midrule
        \rowcolor[HTML]{EFEFEF}
            开发人员 & 可以熟练运用命令行或终端, 可以理解产品的代码及架构, 看完文档之后便可以运用本项目. \\
            编程入门者 & 可以初步使用命令行, 看完文档之后便可以尝试着在脱离 \gui\ 的前提下使用本项目. \\
            无编程经验者 & 没有接触编程, 在看过文档之后不能运用本项目, 此时应提供批处理脚本或 \gui, 可以适当提供线下辅助的使用学习. \\
        \bottomrule
    \end{tabularx}
\end{center}


\subsection{操作环境}
本项目的本地版本主要采用可跨平台的 \TeX, \python\ 编写, 所以可以运行在主流操作系统中, 相应地需要配置编译环境, 具体可参考 \href{https://tex.readthedocs.io}{\readthedocs} 及 \href{https://wiki.python.org/moin/BeginnersGuide/Download}{\python} 官网进行配置. 而在线版本则只需要有可以联网并且安装好了浏览器即可.


\subsection{产品设计及其实现中的约束}
产品设计见第~\ref{S:introduction}~节及第~\ref{S:overall-description}~节. 实现中的约束包括: \begin{enumerate*}[label=\textbf{(\arabic*)}]
    \item 本地版本用户可能无法配置好 \TeX, \python\ 环境;
    \item 本地版本用户可能对数据库的修改存在误差, 导致结果出现错误, 但是本项目中前期不采用捕获异常操作, 可能导致错误不易理解;
    \item 在线版本用户可能设置权限出现问题, 导致私有题库外泄.
\end{enumerate*}


\subsection{用户文档}
目前本项目尚未作出初步版本, 请先参考本文档, 等到项目完成后, 将替代此部分.


\subsection{假设与依赖}
假设用户可以遵循数据库的设计, 可以参考文档配置出高质量的模板文件. 本地版本的成功运行依赖于以上对用户的假设, 而在线版则需考虑服务器安全问题, 假设用户会制造出不遵循规范的数据, 所以在写入数据库之前需要对数据进行严格的校验.



\section{外部接口的要求}\label{S:external-interface-requirements}
\subsection{用户接口}
本地非桌面版本需要考虑用户接口, \python\ 代码需使用 \click\ 库添加命令行使用的参数.目前本项目尚未完善, 所以暂时空缺此小节.


\subsection{硬件接口}
本项目不涉及对硬件进行操作, 所以无硬件接口.


\subsection{软件接口}
所有的代码需要遵循 \href{https://www.python.org/dev/peps/pep-0008/}{\pepeight} 协议, 但是为确保向前兼容, 故不推荐使用类型推导式 (即 \mintinline{python}{num: int = 1} 与 \mintinline{python}{def f() -> int: return int()}), 此外所有的接口需参照以下函数的定义 (\python):
\begin{minted}[breaklines,breakanywhere,linenos,frame=single]{python}
def api_name_must_be_explicit(arg_one, arg_two='default_value', *args, **kwargs):
    '''API name must be explicit, this is a summary of this function.

    :Argument:
        - arg_one: <type>, <explanation>
        - arg_two: str, <explanation>, default is 'default_value'
        - args: list, <explanation>
        - kwargs: dict, <explanation>

    :Return:
        - <case_one>, <explanation>
        - <case_two>, <explanation>
        - <...>

    :Example:
        >>> api_name_must_be_explicit(None)

    :Reference:
        - [<ref_name>](<ref_link>)
    '''
    <function_definition>
\end{minted}


\subsection{通信接口}
本项目的本地版本不涉及通信, 只有在线版本提供通信接口. 目前在线版本尚未完善, 所以暂时空缺此小节.



\section{系统功能}\label{S:system-features}
% 添加系统功能时取消注释此小节并完善内容
% This template illustrates organizing the functional requirements for the
% product by system features, the major services provided by the product. You may
% prefer to organize this section by use case, mode of operation, user class,
% object class, functional hierarchy, or combinations of these, whatever makes the
% most logical sense for your product.
% \subsection{系统特性一}
% Don't really say ``System Feature 1.'' State the feature name in just a few
% words.
% \subsubsection{说明及其优先级}
% Provide a short description of the feature and indicate whether it is of
% High, Medium, or Low priority. You could also include specific priority
% component ratings, such as benefit, penalty, cost, and risk (each rated on a
% relative scale from a low of 1 to a high of 9).

% \subsubsection{灵感来源}
% List the sequences of user actions and system responses that stimulate the
% behavior defined for this feature. These will correspond to the dialog elements
% associated with use cases.

% \subsubsection{功能性需求}
% Itemize the detailed functional requirements associated with this feature.
% These are the software capabilities that must be present in order for the user
% to carry out the services provided by the feature, or to execute the use case.
% Include how the product should respond to anticipated error conditions or
% invalid inputs. Requirements should be concise, complete, unambiguous,
% verifiable, and necessary. Use ``TBD'' as a placeholder to indicate when necessary
% information is not yet available.

% Each requirement should be uniquely identified with a sequence number or a
% meaningful tag of some kind.

% \begin{itemize}
%     \item REQ-1:
%     \item REQ-2:
% \end{itemize}



\section{系统其他非功能需求}\label{S:other-nonfunctional-requirements}
\subsection{系统性能需求}
鉴于家用笔记本可以处理本项目本地版本的计算量 (最大的运算量分配在 \TeX\ 编译上), 所以在项目构思阶段对系统的性内没有要求. 但是在线版本需要考虑系统的性能. 假设在线版本供一个学校使用, 则需要使用高效的服务器架构 (例如 \nginx), 如果经费有限, 则考虑使用排队策略处理请求.


\subsection{系统安全需求}
系统安全方面的问题交由专业人员处理, 目前能做的是服务器关闭不用端口, 设置长密码及申请 \https\ 安全证书.


\subsection{系统保密需求}
为保证用户数据的保密性, 则在线版本的数据库存储需进行加密. 常规来讲, 用户的所有数据 (尤其密码) 更需要进行加密处理.


\subsection{系统质量指标}
衡量系统的质量主要有以下三个指标: \begin{enumerate*}[label=\textbf{\arabic*}]
    \item 本地及在线版本的编译时间;
    \item 服务器处理请求每秒的数量;
    \item 在线版本的异常捕获能力.
\end{enumerate*}


\subsection{授权规则}
\subsubsection{教育授权}
教育授权 (目标为学校) 有待商榷.

\subsubsection{商业授权}
商业授权 (目标为教育机构) 有待商榷.



\section{附录}\label{S:appendix}
% 添加附录时取消注释此小节并完善内容
% Define any other requirements not covered elsewhere in the SRS. This might
% include database requirements, internationalization requirements, legal
% requirements, reuse objectives for the project, and so on. Add any new sections
% that are pertinent to the project.
% \subsection{附录 A: 词汇表}
% %see https://en.wikibooks.org/wiki/LaTeX/Glossary
% Define all the terms necessary to properly interpret the SRS, including
% acronyms and abbreviations. You may wish to build a separate glossary that spans
% multiple projects or the entire organization, and just include terms specific to
% a single project in each SRS.

\end{document}
