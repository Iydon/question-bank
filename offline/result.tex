% !Mode:: "TeX:UTF-8"
% !TEX program  = xelatex
\newcommand{\学校全称}{南方科技大学}
\newcommand{\学校英文}{Southern University of Science and Technology}
\newcommand{\学校简称}{SUSTech}
\newcommand{\学年学期}{2019--2020学年秋季学期}
\newcommand{\科目全称}{计算金融}
\newcommand{\科目代码}{MA216}
\newcommand{\试卷类别}{A}
\newcommand{\考试方式}{闭卷}
\newcommand{\打印纸张}{a4paper}
\newcommand{\展示答案}{false}

\documentclass{exam}
% 使用 XeLaTeX 或者 LuaLaTeX 编译.
\usepackage{ifxetex,ifluatex}
\usepackage{ifthen}
	\ifthenelse{\boolean{xetex}\OR\boolean{luatex}}{}{
		\errmessage{You Should Use XeLaTeX or LuaLaTeX To Compile.}}
% 配置选项.
\usepackage{geometry}
	\geometry{layout=\打印纸张}
\usepackage{xstring}
	\IfStrEq{\展示答案}{true}{\printanswers}{}
% 使用中文, 颜色, 数学符号等.
\usepackage{ctex,xcolor,mathtools,amssymb}
% Exam 文类选项
	% 提示语配置
	\renewcommand{\solutiontitle}{\noindent\textbf{解答:}}
	% 正确答案样式
	\CorrectChoiceEmphasis{\color{black}\bfseries}
	% 分数计量
	\pointpoints{分}{分} % 单复数
	\bonuspointpoints{分 (附加)}{分 (附加)} % 附加分单复数
	% 宏与环境
	\newenvironment{不定项选择}%
		{\begin{oneparchoices}}%
		{\end{oneparchoices}}
	\newenvironment{纵向不定项选择}%
		{\begin{choices}}%
		{\end{choices}}
	\newcommand{\填空}[1]{\fillin[#1]}
	\newcommand{\判断正误}[1]{\fillin[#1][0.25in]}
	\newenvironment{解答}%
		{\begin{solutionorbox}}%
		{\end{solutionorbox}}
	\newenvironment{解析}%
		{\renewcommand{\solutiontitle}{\noindent\textbf{解析:}}\begin{solutionorbox}}
		{\renewcommand{\solutiontitle}{\noindent\textbf{解答:}}\end{solutionorbox}}
% 用户自定义.

\begin{document}
\begin{questions}

% 本题考察了模板的使用.
\question[5] 本题为不定项选择.\par
\begin{不定项选择}
\CorrectChoice 正确答案-4
\CorrectChoice 正确答案-3
\choice 错误答案-1
\choice 错误答案-2
\end{不定项选择}
\begin{解析}
这里为解析部分.
\end{解析}


%
\question[5] 本题为不定项选择.\par
\begin{不定项选择}
\choice 错误答案-1
\CorrectChoice 正确答案-4
\CorrectChoice 正确答案-3
\choice 错误答案-2
\end{不定项选择}



% 本题考察了模板的使用.
\question[5] 本题为不定项选择.\par
\begin{纵向不定项选择}
\choice 错误答案-1
\CorrectChoice 正确答案-4
\choice 错误答案-2
\CorrectChoice 正确答案-3
\end{纵向不定项选择}
\begin{解析}
这里为解析部分.
\end{解析}


%
\question[5] 本题为不定项选择.\par
\begin{纵向不定项选择}
\CorrectChoice 正确答案-3
\choice 错误答案-2
\choice 错误答案-1
\CorrectChoice 正确答案-4
\end{纵向不定项选择}


\end{questions}
\end{document}